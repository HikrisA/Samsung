Durante la investigación que se realizó para este trabajo se encontraron diversos artículos en los cuales se mencionan algunas de las principales causas del cáncer del pulmón, aunque algunos mencionan información relevante, como el factor de riesgo que presentan el alcohol y el tabaco también encontramos información sobre cáncer de pulmón en personas que jamás han fumado. Esta diversidad en las causas que se discuten nos llevo a aplicar los métodos aprendimos en el \textit{Samsung Innovation Campus} para averiguar si podíamos agrupar a los pacientes en base al nivel de su enfermedad con las características recopiladas.

En la sección de resultados podemos apreciar que nuestro objetivo no es posible, al menos con los datos y técnicas que trabajamos, en la matriz de confusión y el reporte de clasificación para \textit{DBSCAN} se aprecia que el modelo clasifica casi a todos los pacientes con un nivel bajo de la enfermedad lo cual ocasiona que tenga una precisión sumamente baja (36\%). 

Si bien, para el modelo de agrupamiento con \textit{k-means} se observa una mejoría significativa, sigue sin ser adecuada, pues en particular el modelo agrupa de manera muy pobre a los pacientes con un nivel alto de cáncer de pulmón, y esto es un elemento de suma importancia, además de que el agrupamiento para los pacientes con niveles bajos y medios de cáncer tampoco es lo suficientemente preciso. Aunque se podría considerar que el modelo tiene cierta utilidad, puesto que el recall para el nivel \textit{Low} fue de 76\% lo que nos indica que el 76\% de los pacientes que tienen un nivelo bajo de cáncer de pulmón fueron correctamente clasificados, esto sigue estando lejos de una precisión aceptable, y en lo general, la precisión del agrupamiento con \textit{k-means} (67\%) nos permite descartarlo. 