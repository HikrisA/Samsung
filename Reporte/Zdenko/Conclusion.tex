El cáncer de pulmón representa un reto crítico para la salud pública tanto en México como a nivel mundial. Aunque en México ocupa el séptimo lugar en cuanto a frecuencia, se le reconoce como el tumor más letal, siendo la principal causa de muerte por cáncer. De acuerdo con el Dr. Omar Macedo Pérez, oncólogo del Instituto Nacional de Cancerología (INCan), anualmente fallecen cerca de ocho mil mexicanos por esta enfermedad, y se registran alrededor de nueve mil casos nuevos, de los cuales el 85\% están relacionados con el consumo de tabaco.

Este panorama se refleja también a nivel internacional. En la última década, la incidencia del cáncer de pulmón ha incrementado en un 30\%, con más de 2 millones de casos nuevos estimados tan solo en el año 2020 y alrededor de 1.8 millones de muertes. Si esta tendencia continúa, se espera que para el año 2030 se reporten más de 2.7 millones de casos nuevos anualmente. Estas cifras posicionan al cáncer de pulmón como una de las enfermedades oncológicas más agresivas y demandantes de atención prioritaria.

En el contexto mexicano, según datos recientes, en 2020 se registraron 7,811 nuevos casos y 6,733 muertes atribuibles a esta enfermedad. Tales datos refuerzan la relevancia del presente trabajo y la necesidad de contar con herramientas predictivas precisas que permitan anticipar el comportamiento de esta enfermedad en el futuro.

A partir del desarrollo de este proyecto y con base en la metodología aplicada para los modelos de predicción ---específicamente, la regresión lineal y la red neuronal recurrente (LSTM)--- se puede afirmar que los resultados obtenidos son coherentes con las estadísticas actuales. Las predicciones realizadas por ambos modelos se aproximan significativamente a las cifras reales reportadas en el país, lo que respalda su utilidad y validez para escenarios futuros.

\begin{enumerate}
    \item La evolución de las muertes por cáncer de pulmón en México sigue una tendencia lineal ascendente, reflejando un incremento sostenido año tras año. Esta tendencia podría agravarse en el futuro debido al creciente consumo de productos alternativos que contienen nicotina, como los vapeadores electrónicos y los sobres de nicotina (snus), los cuales representan un riesgo emergente para la salud pública.
    
    \item El modelo de regresión lineal mostró un buen desempeño en términos de precisión, aprovechando la estructura temporal de los datos disponibles. No obstante, al incrementar el volumen de datos y al incorporar variables adicionales relevantes ---como edad, género, comorbilidades, o hábitos de consumo---, las redes neuronales recurrentes podrían superar el rendimiento de los modelos tradicionales, al capturar patrones no lineales y relaciones más complejas en la evolución de la mortalidad.
\end{enumerate}

Estos hallazgos demuestran el potencial de las técnicas de aprendizaje automático y profundo como herramientas valiosas en el análisis epidemiológico y la toma de decisiones en salud pública. Su implementación puede contribuir significativamente al diseño de políticas preventivas y a la asignación más efectiva de recursos sanitarios en el país. \\ 